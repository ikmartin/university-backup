\documentclass[hidelinks,11pt,dvipsnames]{article}

% font stuff
\linespread{1.1}

% xcolor commonly causes option clashes, this fixes that
\PassOptionsToPackage{dvipsnames,table}{xcolor}
\usepackage[tmargin=1in, bmargin=1in, lmargin=0.8in, rmargin=1in]{geometry}


% use the better epsilon
\renewcommand{\epsilon}{\varepsilon}

% suppress the warning page
\pdfsuppresswarningpagegroup=1

% enable synctex for inverse search
\synctex=1

% include packages
\usepackage{macrosabound,homework,math-env,quiver}
\usepackage{microtype}

% bibtex stuff
\usepackage[backend=biber,style=alphabetic,sorting=anyt]{biblatex}
\addbibresource{main.bib}

% colored text shortcuts
\newcommand{\blue}[1]{\color{MidnightBlue}{#1}}
\newcommand{\red}[1]{\textcolor{Mahogany}{#1}}
\newcommand{\green}[1]{\textcolor{ForestGreen}{#1}}

% use mathptmx pkg while using default mathcal font
\DeclareMathAlphabet{\mathcal}{OMS}{cmsy}{m}{n}

% fixes the positioning of subscripts in $$ $$
\renewcommand{\det}{\operatorname{det}}

\usetikzlibrary{positioning, arrows.meta}
\newcommand{\here}[2]{\tikz[remember picture]{\node[inner sep=0](#2){#1}}}



\begin{document}
\pagestyle{empty}
	\LARGE
\begin{center}
	Elliptic Curves Example Sheet 2\\
	\Large
	Isaac Martin \\
    Last compiled \today
\end{center}
\normalsize
\vspace{-2mm}
\hru

\begin{homework}[e]
	\prob
	\prob Let $A$ be an abelian group and let $q:A\to \bZ$ be a map satisfying
	\begin{align*}
		q(x+y)-q(x-y) = 2q(x)+2q(y).
	\end{align*}
	Prove that $A$ is a quadratic form.
	\begin{prf}
		Recall that to be a quadratic form, $q$ must satisfy
		\begin{enumerate}[(i)]
			\item $q(nx) = n^2q(x)$ for all $x \in A$ and $n \in \bZ$ 
			\item $\langle x,y\rangle = q(x+y)-q(x)-q(y)$ is a $\bZ$-bilinear pairing.
		\end{enumerate}
        We prove these properties by induction.
		
		\bigskip

        \noindent \emph{(i)}~ Notice that $q(1\cdot x) = 1^2q(x)$ trivially, $q(0+0) + q(0-0) = 2q(0)+2q(0)$ so $q(0) = 0$, and $q(2x) = 2q(x) + 2q(x) - q(x - x) = 4q(x)$ for all $x\in A$; hence, (i) holds for $n = 0,1$ and $2$. Now suppose that (i) holds for all positive values $k$ with $n > k > 2$. By induction,
		\begin{align*}
			q(nx) &= 2q((n-1)x) + 2q(x) - q((n-2)x) \\
				  &= 2(n-1)^2q(x) + 2q(x) - (n-2)^2q(x) \\
				  &= (2n^2 - 4n + 2 + 2 - n^2 + 4n - 4)q(x) = n^2q(x),
		\end{align*}
		so (i) holds for all values $n \geq 0$. 

		Finally, if $n \geq 0$ then
		\begin{align*}
			q(-nx) &= q(x - (n+1)x) = 2q(x)+2q((n+1)x) - q(x+(n+1)x) \\
				   &= 2q(x) + 2(n+1)^2q(x) - (n+2)^2q(x) \\
				   &= (2 + 2n^2+4n+2 - n^2 - 4n - 4)q(x) = n^2 q(x).
		\end{align*}
		This means $q(nx) = n^2q(x)$ for all $n \in \bZ$ and $x \in A$.
		
		\bigskip

		\noindent \emph{(ii)}~ Since the pairing $\langle x,y\rangle$ is invariant under the permutation $x \mapsto y$ and $y \mapsto x$, it suffices to prove that $\langle -,-\rangle$ is $\bZ$ linear in the first coordinate, i.e. that $\langle nx,y\rangle n\langle x,y\rangle$ for all $n \in \bZ$ and $x,y \in A$. We first treat the case that $n \geq 0$. This induction argument requires that the statement hold true for $n-1,n-2$ and $n - 3$, so we need the cases that $n = 0, 1$ and $2$ before proceeding to the induction step.

		\noindent \hspace{1em}\underline{$n = 0$}:~ $\langle 0 \cdot x,y\rangle = q(0\cdot x + y) - q(0\cdot x) - q(y) = q(y) - q(y) = 0 = 0\cdot \langle x,y\rangle$.

		\noindent \hspace{1em}\underline{$n = 1$}:~ This is trivially satisfied.

		\noindent \hspace{1em}\underline{$n = 2$}:~ We invoke the equality $q(2x) = 4q(x)$ provided by (i) here.
		\begin{align*}
			\langle 2x,y\rangle &= q(2x + y) - q(2x) - q(y) \\
				&= q(x + (x+y)) - q(2x) - q(y) \\
				&= 2q(x) + 2q(x+y) - q(x - (x+y)) - 4q(x) - q(y) \\
				&= 2q(x+y) - 2q(x) - q(-y) - q(y) \\
				&= 2\big(q(x+y) -q(x) - q(y)\big) = 2\langle x,y\rangle.
		\end{align*}
        
		Assume now that $n > 2$ and that $\langle kx,y\rangle = k\langle x,y\rangle$ holds for $n > k\geq 0$. This means
		\begin{align*}
			\langle kx,y\rangle = q(kx+y) - q(kx) - q(y) = k\big(q(x+y) - q(x) - q(y)\big)
		\end{align*}
		for $0 \leq k < n$ and so
		\begin{align*}\tag{$\ast$}\label{eqn:problem2-(ii)}
			q(kx + y) &~=~ k\big(q(x+y) - q(x) - q(y)\big) + k^2q(x) + q(y) \\
					  &~=~ kq(x+y) + (k^2 - k)q(x) - (k-1)q(y).
		\end{align*}
		We can now prove the desired statement:
		\begin{align*}
			\langle nx,y\rangle &~=~ q(nx + y) - q(nx) - q(y) \\
			  &~=~ 2q(x) + 2q\big((n-1)x + y\big) - q\big((n-2)x + y\big) - q(nx) - q(y) \\
			  &~=~ 2q(x) + \overbrace{2(n-1)(q(x+y)+2(n-1)(n-2)q(x) - 2(n-2)q(y))}^{2q\big((n-1)x + y\big) \text{ by (\ref{eqn:problem2-(ii)}) }} \\
			  &\hspace{1.5em} - \overbrace{(n-2)q(x+y) - (n-2)(n-3)q(x) + (n-3)q(y)}^{-q\big((n-2)x+y\big) \text{ by (\ref{eqn:problem2-(ii)})}} - n^2q(x) - q(y) \\
			  &~=~ \big(2(n-1) -(n-2)\big)q(x+y) +\big(2 + 2(n-1)(n-2)-(n-2)(n-3)-n^2\big)q(x) \\
			  &\hspace{1.5em}+ \big(-2(n-2)+(n+3)-1\big)q(y) \\
			  &~=~ nq(x+y) - nq(x) - nq(y) \\
			  &~=~ n\langle x,y\rangle.
		\end{align*}
        
		We now must treat the case that $n < 0$. If $n = -1$ we get
		\begin{align*}
			\langle -x,y\rangle &~=~ q(-x + y) - q(-x) - q(y) \\
			  &~=~ 2q(x) + 2q(y) - q(x+y) - q(x) - q(y) = -\langle x,y\rangle
		\end{align*}
		without too much trouble. Using this together with the $n\geq 0$ case gives us
		 \begin{align*}
			\langle -nx,y\rangle = - \langle nx,y\rangle
		\end{align*}
	    for $n \geq 0$, so we conclude that $\langle nx,y\rangle = n\langle x,y\rangle$ for all $n \in \bZ$ and are done.
	\end{prf}
	\prob Find a translation-invariant differential $\omega$ on the multiplicative group $\bG_m$. Show that if $[n]:\bG_m \to \bG_m$ is the endomorphism $x \mapsto x^n$ then $[n]^*\omega = n\omega$.
	\begin{prf}
		An invariant differential of a formal group law $F \in R\llbracket X,Y\rrbracket$ is a differential form
		\begin{align*}
			\omega = P(T)dT \in R\llbracket T\rrbracket dT
		\end{align*}
		which satisfies
		\begin{align*}
			\omega \circ F(&T,S) = \omega(T) \\
						   &\iff \\
			P(F(T,S))&F_X(T,S) = P(T)
		\end{align*}
		where $F_X(T,S)$ is the partial derivative of $F$ in the first variable. The formal group law of $\bG_m$ is $F(X,Y) = X + Y + XY = (1 + X)(1 + Y) -1$, and its partial derivative in $X$ is $F_X(X,Y) = 1 + Y$. We are therefore looking for some $P(T) \in R\llbracket T\rrbracket$ such that
		\begin{align*}
			P((1+T)(1+S)-1) \cdot (1+S) = P(T).
		\end{align*}
		It is fortunate that we discussed the element $\frac{1}{1 - X} = 1 + x + x^2 + ... \in R\llbracket T\rrbracket$ in class -- a slight modification, the power series $P(T) = \frac{1}{1 + T} = 1 - T + T^2 - T^3 + ... \in R\llbracket T\rrbracket$, will do the trick:
		\begin{align*}
			P((1+T)(1+S)-1)\cdot (1 + S) = \frac{1}{(1+T)(1+S) - 1 + 1}\cdot (1 + S) = \frac{1}{1 + T} = P(T).
		\end{align*}
		Hence the differential form $\omega = \frac{1}{1 + T}$ is an invariant differential of the multiplicative formal group law.
	\end{prf}
\end{homework}

\end{document}
