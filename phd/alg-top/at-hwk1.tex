\documentclass[hidelinks,11pt,dvipsnames]{article}

% font stuff
\linespread{1.1}

% xcolor commonly causes option clashes, this fixes that
\PassOptionsToPackage{dvipsnames,table}{xcolor}
\usepackage[tmargin=1in, bmargin=1in, lmargin=0.8in, rmargin=1in]{geometry}


% use the better epsilon
\renewcommand{\epsilon}{\varepsilon}

% suppress the warning page
\pdfsuppresswarningpagegroup=1

% enable synctex for inverse search
\synctex=1

% include packages
\usepackage{macrosabound,homework,math-env,quiver}
\usepackage{microtype}

% bibtex stuff
\usepackage[backend=biber,style=alphabetic,sorting=anyt]{biblatex}
\addbibresource{main.bib}

% colored text shortcuts
\newcommand{\blue}[1]{\color{MidnightBlue}{#1}}
\newcommand{\red}[1]{\textcolor{Mahogany}{#1}}
\newcommand{\green}[1]{\textcolor{ForestGreen}{#1}}

% use mathptmx pkg while using default mathcal font
\DeclareMathAlphabet{\mathcal}{OMS}{cmsy}{m}{n}

% fixes the positioning of subscripts in $$ $$
\renewcommand{\det}{\operatorname{det}}

\usetikzlibrary{positioning, arrows.meta}
\newcommand{\here}[2]{\tikz[remember picture]{\node[inner sep=0](#2){#1}}}



\def\sset{\subseteq}
\def\iso{\cong}
\def\gend#1{\langle #1\rangle}

\begin{document}
\pagestyle{empty}
	\LARGE
\begin{center}
	Algebraic Topology Homework 0 \\
	\Large
	Isaac Martin \\
    Last compiled \today
\end{center}
\normalsize
\vspace{-2mm}
\hru
\begin{homework}[e]
	\prob Construct an explicit deformation retraction of the torus with one point delted onto a graph consisting of two circles intersecting in a point, namely, longitude and meridian circles of the torus.

	\prob $ $
	\begin{enumerate}[(a)]
		\item Show that the composition of homotopy equivalences $X\to Y$ and $Y\to Z$ is a homotopy equivalence $X\to Z$. Deduce that homotopy equivalence is an equivalence relation.
		\item Show that the relation of homotopy among maps $X \to Y$ is an equivalence relation.
	\end{enumerate}

	\prob Show that a retract of a contractible space is contractible.
	\prob Show that $S^\infty$ is contractible.
	\prob $ $
	\begin{enumerate}[(a)]
		\item Show that the mapping cylinder of every map $f:S^1 \to S^1$ is a CW complex.
		\item Construct a 2-dimensional CW complex that contains both an annulus $S^1\times I$ and a M\"obius band as deformation retracts.
	\end{enumerate}
	\prob Show that a CW complex is contractible if is the union of two contractible subcomplexes whose intersection is also contractible.
	\prob Use Corollary 0.20 to show that if $(X,A)$ has the homotopy extension property, then $X\times I$ deformation retracts to $X\times \{0\}\cup A\times I$. Deduce from this that Proposition 0.18 holds more generally for any pair $(X_1,A)$ satisfying the homotopy extension property.
\end{homework}
\end{document}
