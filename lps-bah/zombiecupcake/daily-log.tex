\documentclass[hidelinks,11pt,dvipsnames]{article}

% font stuff
\linespread{1.1}

% xcolor commonly causes option clashes, this fixes that
\PassOptionsToPackage{dvipsnames,table}{xcolor}
\usepackage[tmargin=1in, bmargin=1in, lmargin=0.8in, rmargin=1in]{geometry}


% use the better epsilon
\renewcommand{\epsilon}{\varepsilon}

% suppress the warning page
\pdfsuppresswarningpagegroup=1

% enable synctex for inverse search
\synctex=1

% include packages
\usepackage{macrosabound,homework,math-env,quiver}
\usepackage{microtype}

% bibtex stuff
\usepackage[backend=biber,style=alphabetic,sorting=anyt]{biblatex}
\addbibresource{main.bib}

% colored text shortcuts
\newcommand{\blue}[1]{\color{MidnightBlue}{#1}}
\newcommand{\red}[1]{\textcolor{Mahogany}{#1}}
\newcommand{\green}[1]{\textcolor{ForestGreen}{#1}}

% use mathptmx pkg while using default mathcal font
\DeclareMathAlphabet{\mathcal}{OMS}{cmsy}{m}{n}

% fixes the positioning of subscripts in $$ $$
\renewcommand{\det}{\operatorname{det}}

\usetikzlibrary{positioning, arrows.meta}
\newcommand{\here}[2]{\tikz[remember picture]{\node[inner sep=0](#2){#1}}}



% lecture commands
\newcounter{entry-counter}
\newcommand{\entry}[1]
{
	\newpage
	\addtocounter{entry-counter}{1}
    \tchap{Entry \arabic{entry-counter}}
	\vspace{-1.5em}
    \begin{center}
		\small \emph{Written: #1}
    \end{center}
}

% start document
\begin{document}
\pagestyle{empty}
	\LARGE
\begin{center}
	Sporadic Notes Taken While at LPS Probabilistic Computing \\
	\Large
	Isaac Martin \\
    Last compiled \today
\end{center}
\normalsize
\vspace{-2mm}
\hru
\tableofcontents

\entry{2022-June-24, Friday}

\section{Reverse Ising Model Description}
The Ising model is a model of ferromagnetism. We care more about the model itself and less about its physical properties, so here we provide an abstraction. 

\begin{defn}\label{defn:ising-graph}
	Suppose you are given an undirected graph $G$ with vertices and edges decorated as follows. 
	\begin{itemize}
			\item Vertex $i$ receives a \emph{spin} $s_i \in \{\pm 1\}$ and a \emph{local bias} $h_i \in \bR$. The former controls the direct of a vertex's intrinsic contribution to the energy of the system while the second controls the weight of the contribution.
			\item Edge $e_{ij}$ is given a \emph{coupling strength} $J_{ij} \in \bR$. If edge $e_{ij}$ does not exist then $J_{ij} = 0$ by convention.
	\end{itemize}
	Call such a graph an \emph{Ising graph} and call the verticies \emph{sites}. The total energy of the system is given by the Hamiltonian
	\begin{align*}
		H(G) = \sum_{i \in V} h_is_i ~+~ \sum_{i,j \in V} J_{ij}s_j.
	\end{align*}
\end{defn}

The \textbf{forward Ising problem} takes as input the values of the local biases $\{h_i\}$ and the coupling strengths $\{J_{ij}\}$ and then outputs the (or rather, ``a'') tuple $(s_1,...,s_n)$ of spins which minimize the Hamiltonian.

The \textbf{reverse Ising problem} takes as input a tuple of spins and then outputs the set of $h_{i}$ and $J_{ij}$ which minimize the Hamiltonian, if possible. It is of interest as a slight modification of this problem would allow for the construction of logic gates from Ising graphs, in which a spin value of $1$ and $-1$ represent true and false respectively.



\begin{example}\label{example:and-gate}
	
\end{example}

\end{document}
