\documentclass[raggedright, nofonts, notitlepage, openany, debug]{tufte-book}

%%% Sets numbering depth to section level (e.g, no numbered subsections)
\setcounter{secnumdepth}{1}

% include notes style file from Abhishek Shivkumar
\usepackage{notes,macrosabound}

% make font smaller
\usepackage[fontsize=9pt]{fontsize}
% set margins
% only use if strictly necessary, tufte-book calls geometry already
% double geometry call results in "overspecified margins" warning if done as below
% \geometry{left=1in, right=3in, top = 1in, bottom=1in}

%%%%%%%%%%%%%%%%%%
%%%% FOR DEBUGGING
%%%%%%%%%%%%%%%%%%
%\usepackage{layout}
%\usepackage{showframe}

%%% Removes paragraph indentation and changes paragraph line skip
\makeatletter
\renewcommand{\@tufte@reset@par}{%
  \setlength{\RaggedRightParindent}{0pc}%
  \setlength{\JustifyingParindent}{0pc}%
  \setlength{\parindent}{0pc}%
  \setlength{\parskip}{9pt}%
}
\@tufte@reset@par
\makeatother

\usepackage{natbib}
\bibliographystyle{unsrtnat}

% ensures that the references show up as an unnumbered section
\def\bibsection{\section*{\refname}} 
\begin{document}
%%% The Title and Author only need to be set once at the start of the document. If you take notes for multiple courses in the same document (for example, in a multi-semester sequence for the same course), you can separate the courses with a new Part, and the semester, lecturer, and course only need to be set once at the start of the new course.
\newpage
\title{Brief Review of ``The Chern Character of the Verlinde Bundle''}
\author{Isaac Martin}
\maketitle
\chapter{Some Background On Chern Characters, CohFTs and the Verlinde Bundle}
\section{Introduction}
This document is meant to provide a bit of exposition on the objects involved in computation of the Chern character of the Verlinde bundle over the moduli space of stable curves, performed by Marian, Oprea, Pandharipande, Pixton and Zvonkine \cite{marian2016chern}. More precisely, when I first attempted to read this paper, I found I lacked the requisite knowledge of cohomological field theories and Chern classes to understand the results. This is my attempt to gather the necessary definitions in one place and to provide context for how they first together in Marian et. al.'s paper.

\section{Basic Structure of the Moduli Space of Stable Curves}
I had encountered $\ol{\cM}_{g,n}$ prior to reading the Marian et. al. paper, but was unfamiliar with some of the structure essential to the results. Here, we briefly cover that structure.

Recall that a Riemann surface $C$ \marginnote{Instead of ``Riemann Surface'' we could equivalently say ``smooth curve over $\bC$''} with $n$ distinct marked points $x_1,...,x_n$ is said to be \textbf{stable} if there is only finitely many automorphisms $C\to C$ which fix each $x_i$. A Riemann surface of genus $g$ with $n$ marked points is stable if and only if $2g - 2 + n > 0$. \marginnote{This stability condition comes from the fact that the canonical bundle of $C$ twisted by the marked points is ample if and only if $2g - 2 + n > 0$. Ampleness of the canonical bundles imply that there are only finitely many automorphisms fixing the marked points.} By $\cM_{g,n}$ we denote the course moduli space of all genus $g$ Riemann surfaces over $\bC$ with $n$ marked points, and by $\ol{\cM}_{g,n}$ the Deligne-Mumford compactification of $\cM_{g,n}$. It can be shown that $\ol{\cM}_{g,n}$ consists of connected curves with at worst nodal singularities. \marginnote{An element in $\ol{\cM}_{g,n}$ is called a \emph{stable curve}, and is denoted $(C,x_1,...,x_n)$.}

\begin{rmk}
  There is an obvious action of $S_n$ on $\mgnbar_{g,n}$ given by permuting the marked points. This corresponds to the first axiom of CohFTs.
\end{rmk}

Suppose now that we have two stable curves $(C_1,x_1,...,x_{n_1+1}) \in \ol{\cM}_{g_1,n_1}$ and $(C_2,y_1,...,y_{n_2+1}) \in \ol{\cM}_{g_2, n_2}$. We can obtain a stable curve $$(S(C_1, C_2),x_{1..n_1}, y_{1..n_2})$$ of genus $g = g_1 + g_2$ with $n = n_1 + n_2$ marked points by defining
\begin{align*}
  S(C_1,C_2) = C_1 \sqcup C_2/\sim
\end{align*}
where $\sim$ identifies $x_{n_1 + 1}$ and $y_{n_2 + 1}$ while leaving all else fixed. This gives us a continuous map
\begin{align*}
  s: \mgnbar_{g_1,n_1 + 1} \times \mgnbar_{g_2,n_2+1} \to \mgnbar_{g_1 + g_2,n_1 + n_2},
\end{align*}
which is sometimes called the ```separating sewing map`'' or in the case of Marian et. al. the ``second gluing map''. If there is a second gluing map, there ought to be a first gluing map; this is a smooth function
\begin{align*}
  q: \mgnbar_{g-1,n+2} \to \mgnbar_{g,n}
\end{align*}
which increases the genus of a curve $(C,x_1,...,x_{n+2})$ by identifying $x_{n+1}$ and $x_{n+2}$ while leaving all other points fixed. This is sometimes called the \emph{non-separating gluing map}.
\begin{rmk}
  The two gluing maps are treated by the second axiom of CohFTs.
\end{rmk}

The final bit of structure we wish to highlight is the forgetful map $p:\mgnbar_{g,n+1} \to \mgnbar_{g,n}$ defined
\begin{align*}
  p(C,x_1,...,x_{n+1}) = (C,x_1,...,x_n).
\end{align*}
Notice that if $2g - 2 + (n+1) = 1$, this map will not be well-defined.

\begin{rmk}
  The forgetful map corresponds to the third axiom of CohFTs.
\end{rmk}

\section{Chern Classes and Chern Characters}
The author had not seen Chern characters prior to his presentation, and had encountered Chern classes only in passing. Each is a representation of the same data in slightly different ways. We use this section to review various definitions.

\subsection{Chern Classes}
Chern classes are characteristic classes associated to complex vector bundles, and they have become ubiquitous in both algebraic and complex geometry since their introduction by Shiing-Shen Chern in 1946 (see \cite{chern-class-first-paper}); the $n$th Chern class of a vector bundle $\epsilon \to X$ associates to $\epsilon$ a cohomology class in $H^{2n}(X)$. There are many constructions, but we consider the axiomatic approach found in \cite{complex-vector-bundles-book} to minimize topological prerequisites.

Let $X$ be a smooth scheme over $\bC$. We have four axioms which determine the Chern classes of $\epsilon$.

\textbf{Axiom 1:} For each vector bundle $\epsilon \to X$ and for each $i\geq 0$, the $i$th Chern class satisfies $c_i(\epsilon) \in H^{2i(X;\bC)}$, i.e. it is a degree $2i$ cohomology class. We define $c_0(\epsilon) = 1$.

\begin{rmk}
  We set $c(\epsilon) = \sum_{i=1}^\infty c_i(\epsilon) \in H^\bullet(X;\bC)$ and call it the \emph{total Chern class} of $\epsilon$.
\end{rmk}
\textbf{Axiom 2:} ~(Naturality).~ Let $\epsilon$ be a complex vector bundle over $X$ and let $f:Y\to X$ be a smooth map. Then
\begin{align*}
  c(f^*\epsilon) = f^*(c(\epsilon)) \in H^\bullet(Y;\bC)
\end{align*}
where $f^*$ is the pullback of $\epsilon$ over $Y$.

\textbf{Axiom 3:} ~(Whitney sum formula).~ Let $\epsilon_1,...,\epsilon_q$ be complex line bundles over $X$. Let $\epsilon_1\oplus ...\oplus \epsilon_q$ be their Whitney sum. Then
\begin{align*}
  c(\epsilon_1\oplus ...\oplus \epsilon_q) = c(\epsilon_1) \cdot ...\cdot c(\epsilon_q),
\end{align*}
where the product is the cup product.

\textbf{Axiom 4:} ~(Normalization).~ Let $\cO(-1)$ be the tautological line bundle over $\bP^1_{\bC}.$. Then $-c_1(\cO(-1))$ is the generator of $H^2(\bP_{\bC}^1;\bZ)$, that is, $c_1(\cO(-1))$ integrated on the fundamental 2-cycle $\bP^1_{\bC}$ is equal to $-1$.

\begin{rmk}
  In the case that $\epsilon$ is a complex Hermitian vector bundle over a smooth manifold $X$, the Chern classes appear as the coefficients of the characteristic polynomial of the curvature form of $X$. More explicitly,
  \begin{align*}
    \det\left(\frac{it\Omega}{2\pi} + I\right) = \sum_{i=0}^\infty c_i(\epsilon)t^i.
  \end{align*}
  Note that this implies $c_i(\epsilon) = 0$ whenever $i > \rank \epsilon$, a fact which holds more generally.
\end{rmk}
\subsection{Splitting Principle and the Chern Character}


\section{Cohomological Field Theories}
Here we review the basic definitions of cohomological field theories and the facts needed in the Marian et. al. paper. I learned most of this from S. Chiarello's master's thesis \cite{chiarello2016telemans}. Let $q$ and $s$ denote the first and second gluing maps on $\mgnbar_{g,n}$, as introduced previously, and let $p$ be the forgetful map.

\subsection{Definition}

\begin{defn}
A cohomological field theory (henceforth abbreviated  CohFT) consists of the following data:
\begin{itemize}
  \item a fixed $\bC$-vector space $V$
  \item a distinguished nonzero element $\bfone \in V$
  \item a nondegenerate bilinear form $\eta: V\times V\to \bC$
  \item for each $2g - 2 + n >0$, a $\bC$-linear map
    \begin{align*}
      \Omega_{g,n} : V^{\otimes n}\to H^\bullet(\ol{\cM}_{g,n})
    \end{align*}
    which satisfy three additional axioms corresponding to pieces of the structure of $\mgnbar)_{g,n}$. We denote by $\Omega$ the CohFT defined by this data.
\end{itemize}
\end{defn}
These are the additional axioms:
\begin{enumerate}[(i)]
  \item each $\Omega_{g,n}$ is invariant under pre-composition with the $S_n$ action on $V^{\otimes n}$\marginnote{If $\sigma \in S_n$ is a permutation, then its action on $\mgnbar_{g,n}$ induces a map on cohomology classes $$H^\bullet(\mgnbar_{g,n})\to H^\bullet(\mgnbar_{g,n}).$$ Axiom (i) means that we have \begin{align*}
        \Omega_{g,n}&(v_{\sigma(1)\otimes ...\otimes v_\sigma(n)})\\ &= \sigma^*\Omega_{g,n}(v_1\otimes ...\otimes v_n).
  \end{align*}}
  \item The pullback $q^*\Omega_{g,n}$ equals the contraction of $\Omega_{g-1,n+2}$ with $\eta^{-1}$ at the two extra marked points $x_{n+1}$ and $x_{n_2}$. \marginnote{Here we need that $\eta$ is non-degenerate for $\eta^{-1}$ to behave as we expect.} \marginnote{Axiom (i) is straightforward and is referred to as ``invariance under permutation''. Axioms (ii) and (iii) can be summarized by saying ``$\Omega$ is compatible with gluing maps''. Axiom (iv) says that $\Omega$ is compatible with the forgetful map on $\mgnbar_{g,n}$, where defined.}
  \item The pullback $s^*\Omega_{g_1+g_2,n_1+n_2}$ is equal to the contraction of $\eta^{-1}$ at the two marked points identified by $s$.
  \item For all $v_1,...,v_n\in V$ we get
    \begin{align*}
      \Omega_{g,n+1}(v_1\otimes...\otimes v_n \otimes \bfone) = p^*\Omega_{g,n}(v_1\otimes ... \otimes v_n).
    \end{align*}
    In the special case that $g = 0$ and $n = 3$, in which case $p$ is not defined, we also require
    \begin{align*}
      \Omega_{0,3}(v_1\otimes v_2\otimes \bfone) = \eta(v_1,v_2).
    \end{align*}
\end{enumerate}

\begin{rmk}
  Specifying a map $V^{\otimes n}$ to $H^\bullet(\mgnbar_{g,n})$ is the same as specifying an element of $H^\bullet(\mgnbar_{g,n})\otimes (V^{*})^{\otimes n}$, and indeed, this is how they define CohFTs in Marian et. al. I find the definition given here nicer, however, particularly since axiom (iv) considers the $\Omega_{g,n}$ action on $V^{\otimes n}$.
\end{rmk}

\subsection{Frobenius Algebra Associated to a CohFT}



\newpage
\bibliography{main}
\end{document}
