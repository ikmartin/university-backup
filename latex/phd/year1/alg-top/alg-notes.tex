\documentclass[hidelinks,11pt,dvipsnames]{article}

% font stuff
\linespread{1.1}

% xcolor commonly causes option clashes, this fixes that
\PassOptionsToPackage{dvipsnames,table}{xcolor}
\usepackage[tmargin=1in, bmargin=1in, lmargin=0.8in, rmargin=1in]{geometry}


% use the better epsilon
\renewcommand{\epsilon}{\varepsilon}

% suppress the warning page
\pdfsuppresswarningpagegroup=1

% enable synctex for inverse search
\synctex=1

% include packages
\usepackage{macrosabound,homework,math-env,quiver}
\usepackage{microtype}

% bibtex stuff
\usepackage[backend=biber,style=alphabetic,sorting=anyt]{biblatex}
\addbibresource{main.bib}

% colored text shortcuts
\newcommand{\blue}[1]{\color{MidnightBlue}{#1}}
\newcommand{\red}[1]{\textcolor{Mahogany}{#1}}
\newcommand{\green}[1]{\textcolor{ForestGreen}{#1}}

% use mathptmx pkg while using default mathcal font
\DeclareMathAlphabet{\mathcal}{OMS}{cmsy}{m}{n}

% fixes the positioning of subscripts in $$ $$
\renewcommand{\det}{\operatorname{det}}

\usetikzlibrary{positioning, arrows.meta}
\newcommand{\here}[2]{\tikz[remember picture]{\node[inner sep=0](#2){#1}}}


\usepackage{indentfirst}

% Title page stuff
\title{Notes for Tropical Geometry\\ \vspace{0.5em}{\Large Fall 2022}\vspace{0.5em}\\ The University of Texas at Austin \\ Lectured by Bernd Seibert}
\date{Last Compiled: \today}
\author{Isaac Martin}

% start document
\begin{document}
\pagestyle{empty}
\maketitle
\newpage
\tableofcontents
\newpage
\entry{2022-Aug-26}

Left off last time by defining cell complexes aka CW complexes. Recall that that it is a space $X$ which can be constructed ass a discrete set $X^0$ and then for all $n > 0$ we take $X^n = X^{n-1} \cup D^n_\alpha$ where $\alpha$ index in an index set indexes on $n$-cells.

We're given maps $\varphi_\alpha:S^{n-1} \to X^{n-1}$ with $X^{(n)}$ is $X^{n-1} \cup \coprod D_\alpha ^n$ with identification $x \in S^{n-1}_\alpha$ is identified with $\varphi_\alpha(x) \in X^{n-1}$.

The $\varphi_\alpha$'s need not be injective, but $D_\alpha^n - S^{n-1}_\alpha$ does inject into $X^n$.

We write $e_\alpha$ for the interior of $D^n_\alpha$.

Either $X = X^n$ \textbf{or} $X = \bigcup_n X^n$ with the ``weak topology'': a subset of $X$ is closed if and only if its intersection with each $X^n$ is open.

\begin{example}\label{example:different-sphere-constructions}  
  \underline{Basic Examples}, all ways to write $S^n$:
  \begin{enumerate}[(i)]
    \item $\{\text{norm 1 vectors } in \bR^{n+1}\}$
    \item $D^n / \partial D^n$ meaning crush $S^{n-1} = \partial D^n$
    \item cell complex with $X^n = 1$ pt with an $n$-disk attached (by a unique map $\partial D^n\to \{pt\}$).
    \item 1-pt compactification of $\bR^{n}$
    \item Cell complex: $S^0 = S^{n-1}$ with $2$ disks attached.
  \end{enumerate}
  (5) allows for the definition of $S^\infty = \bigcup_n S^n$, which is not locally compact but is contractible.
\end{example}

Key properties of cell complexes: (approx)
\begin{itemize}
  \item normal
  \item locally contractible
  \item every subset is deformation retractable of some neighborhood of itself
  \item ($\star$) every compact set lies in the union of finitely many cells.
  \item ($\star$) a function on $X$ is continuous if and only if its restriction to each cell is continuous, i.e. $\Phi_\alpha : D_\alpha^n \to X$ followed by $f$ is continuous. Recall that this $\Phi_\alpha$ is the characteristic map of cell $e_\alpha$.
\end{itemize}

\begin{example}\label{example:real-projective-space}
  We may define $\bRP^n = S^{n+1}/\{\pm 1\}$. The cell complex structure is a ``quotient of 2-cells-of-each-dimension'' version of $S^n$. $\bRP^n$ has one cell of each dimension, the ataching map $\partial D^n = S^{n-1}$ being the canonical map $S^{n-1} \to S^{n-1}/\{\pm 1\} = \bRP^n$.

  Each cell has its boundary wrapped twice around $\bRP^{n-1}$ in $\bRP^2$.
\end{example}

The following is a severely important/useful tool. If $X$ a space and $A \subseteq X$, then $(X,A)$ has the homotopy extension property (HEP) if for all maps $F:X\to Y$ and every homotopy $F:A\times I \to Y$ , $F|_A$ to some other map $A \to Y$, there exists an extension to $\tilde{F}:X\times I \to Y$. The idea is this: if you're given a motion of $A$ inside $Y$, then you can drag along with it the points of $X$.

\begin{thm}\label{thm:cw-pairs-have-HEP}
  If $X$ is a CW complex and $A\subseteq X$ is a sub-CW complex then $(X,A)$ has the HEP property.
\end{thm}
\begin{thm}\label{thm:contractible-yields-quotient-homotopy}
  If $(X,A)$ has the HEP and $A$ is contractible, then $X \xrightarrow{q} X/A$ is a homotopy equivalence.
\end{thm}
\begin{prf}
  The clever part is writing down a homotopy inverse $g$. Suppose $f_t:A\times I \to A$ is a contraction $(f_0 = \id_A, f_1 =\text{ const })$. We think of the contraction as a map $f_t:A\times I \to X$, and then use $HEP$ to get $f_t:X\times I\ ot X$. Observe $f_1: X\to X$ sends $A$ to a point, inducing a map $g:X/A\to X$ as $X/A$ is exactly all of $A$ collapsed to a point.

  We need to check that $g$ is an inverse. Consider $X \xrightarrow{q} X/A \xrightarrow{g} X$ is $f_1 \simeq \id_X$ and
  \begin{align*}
    X/A \xrightarrow{g} X \xrightarrow{q} X/A
  \end{align*}
  is $\ol{f}_1$ (function $X/A \to X/A$) induced by $f:X\to X$ $\ol{f}_1 \simeq \ol{f}_0$ since $f_t$ sends $A$ into $A \subseteq X$, hence induces $X/A \to X/A$. The $\ol{f}_t$ are a homotopy between $\olf_0 - \id_{X/A}$ and $\olf_1$ = q\circ g.
\end{prf}


\end{document}
