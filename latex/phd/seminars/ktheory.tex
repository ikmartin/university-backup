\documentclass[11pt, raggedright]{tufte-book}

\setcounter{secnumdepth}{1}
\usepackage{notes2, macrosabound, theorem-env}
% set margins
% only use if strictly necessary, tufte-book calls geometry already
% double geometry call results in "overspecified margins" warning if done as below
% \geometry{left=1in, right=3in, top = 1in, bottom=1in}

%%%%%%%%%%%%%%%%%%
%%%% FOR DEBUGGING
%%%%%%%%%%%%%%%%%%
%\usepackage{layout}
%\usepackage{showframe}

%%% Removes paragraph indentation and changes paragraph line skip
\makeatletter
\renewcommand{\@tufte@reset@par}{%
  \setlength{\RaggedRightParindent}{0pc}%
  \setlength{\JustifyingParindent}{0pc}%
  \setlength{\parindent}{0PC}%
  \setlength{\parskip}{9pt}%
}
\@tufte@reset@par
\makeatother

% ensures that the references show up as an unnumbered section
%\def\bibsection{\section*{\refname}} 
\begin{document}
%%% The Title and Author only need to be set once at the start of the document. If you take notes for multiple courses in the same document (for example, in a multi-semester sequence for the same course), you can separate the courses with a new Part, and the semester, lecturer, and course only need to be set once at the start of the new course.
\newpage
\title{K-Theory Seminar}
\author{Isaac Martin}
\maketitle
\chapter{Week 1: Vector bundles, clutching functions, Grassmanians}
\section{Vector Bundles}
Let's discuss the idea of a vector bundle. Let $X$ be a topological space and suppose to every $x\in X$ we have a vector space $E_x$. A \emph{quasi-vector bundle} is the space $\bigsqcup E_x$ equipped with any topology such that the projection $\bigsqcup E_x \to X$ is continuous.

\begin{defn}\label{defn:vector-bundle}
  A \emph{vector bundle} is $\pi:E\to X$ where each $E_x = \pi^{-1}(X)$ $x\in X$ is a vector space and for each $x\in X$ there exists an open neighborhood $U$ of $x$ and a homeomorphism $\varphi:\pi^{-1}(U) \to U\times E_x$ which restricts to a linear isomorphic $E_{x_0} \to \{x_0\}\times E_x$. This map $\varphi$ is called a local trivialization above $U$.
\end{defn}

\begin{example}\label{ex:vector-bundles}$ $
  \begin{enumerate}[(a)]
    \item if $V$ is a vector space then the projection $p:X\times V\to X$ is a vector bundle. We call this a \emph{trivial vector bundle} since $X$ is a local trivialization for any element $x\in X$.
    \item Consider the $\mathbb Z$ action on $\bR \times \bR$ given by $a\cdot (x,y) = (x + a, (-1)^ay)$. Then
    \begin{align*}
      p_1:\bR\times \bR/\bZ \to \bR/\bZ
    \end{align*}
    where $p_1$ is the projection to the first coordinate is a vector bundle. The total space is the M\"obius band and the base space is the central circle.
  \end{enumerate}
\end{example}
\begin{defn}\label{defn:pullback-square}
  Given a continuous map $f:X\to Y$, if we have a vector bundle $E\to Y$, then we get a vector bundle $f^*E$ over $X$ via pull back:
  \begin{center}
    \begin{tikzcd}
      f^*E \arrow[r] \arrow[d]& E\arrow[d] \\
      X \arrow[r,"f"]& Y
    \end{tikzcd}
  \end{center}
  where $(f^*E)_x = E_{f(x)}$.
\end{defn}
\marginnote{If the map $\varphi$ is of constant rank, i.e. if $\varphi_x$ has the same rank for all $x \in X$, then $\ker \varphi$ and $\coker \varphi$ are both vector bundles over $X$ as well.}
\begin{defn}\label{defn:}
  Suppose we have two vector bundles $\pi:E\to X$ and $\pi':E'\to X$ over the same base space. A \emph{morphism} of vector bundles $\varphi:(\pi,E,X)\to (\pi',E',X)$ is a map $E\to E'$ which both commutes with the projection maps $\pi$ and $\pi'$ and restricts to a linear maps on each fiber: $\varphi_x: E_x\to E_x'$. Notice that $E$ and $E'$ are \emph{not} required to be of the same rank.
\end{defn}


\begin{defn}\label{defn:isomorphism-classes-of-vector-bundles}
  The category of vector bundles with base field $K$ over base $X$ is denoted $\Vect_K(X)$. We denote by $\pi_0(\Vect_K(X))$ the category of vector bundles over $X$ up to isomorphism.
\end{defn}

\subsection{Clutching Construction}
Vector bundles can be defined locally and then glued together to give a global vector bundle. Here is the construction:

\begin{thm}\label{thm:gluing-vector-bundles}
  Let $\{U_\alpha\}_{\alpha\in A}$ be a cover for $X$. Denote by $U_{\alpha\beta} = U_\alpha\cap U_\beta$ and $U_{\alpha\beta\gamma} = U_{\alpha\beta} \cap U_\gamma$ etc. 

  Suppose we have a family of vector bundles $E_\alpha \to U_\alpha$ and isomorphisms $g_{\alpha\beta}:E_\alpha|_{U_{\alpha\beta}}\to E_\beta|_{U_{\alpha\beta}}$. If the gluing maps $g_{\alpha\beta}$ additionally satisfy the cocycle condition $g_{\alpha\gamma}|_{U_{\alpha\beta\gamma}} = g_{\alpha\beta}\circ g_{\beta\gamma}$, then we can glue the local vector bundles $E_\alpha \to U_\alpha$ on the intersections $U_{\alpha\beta}$ to get a total bundle $E\to X$ where
  \begin{align*}
    E = \left.\bigsqcup_{\alpha\in A} E_\alpha \middle / (e_\alpha \sim e_\beta ~ \text{ if }~ g_{\alpha\beta}(e_\alpha) = e_\beta)\right. .
  \end{align*}
\end{thm} \marginnote{This is identical to the gluing construction for sheaves, see Hartshorne Exercise 2.1.13.}

The cocycle condition here exists purely to ensure $e_\alpha \sim e_\beta$ is transitive and is hence an equivalence relation.

\chapter{Week 2: More on Vector Bundles and $K^0$}

Two more things about vector bundles. Given vector bundles $E$ and $E'$ over a space $Y$, we can define
\begin{itemize}
  \item The \textbf{Whitney Sum}: this is defined fiberwise: $(E\oplus E')_x \cong E_x \oplus E'_x$\item The \textbf{tensor product}: $(E\otimes E')_x = E_x\otimes E'_x$
  \item The \textbf{pullback:} if $f:X\to Y$ is continuous, then $f^*E = \{(x,e) \in X\times E ~\mid~ f(x) = p(e)\}$ where $p:E\to X$ is the bundle map. Sections are given $(F^*E)_y = E_{f(x)}$ where $f(x) = y$.
\end{itemize}
\marginnote{I don't know what "varies continuously in morphisms" means, I'll include it later.} More generally\, any endofunctor $F: \Vect \to \Vect$ which varies continuously in morphisms extends to an endofunctor in the category of vector bundles via its action on sections.

\textbf{Examples}

\begin{itemize}
  \item Trivial bundles; these are vector bundles which are trivialized over the whole space. When $M$ is a parallelizable manifold, $TM$ is a trivial bundle. \marginnote{This is often taken to be the definition of parallelizable. Alternatively one can define a manifold to be parallelizable if it has a global frame.}
  \item Tautological vector bundles; $\bRP^n, \bCP^n$, Grassmanians.
  \item Subbundles; $S^1 \times \mathbb R^2$
  \item Normal bundles; $M\hookrightarrow N$ is a smooth embedding, a normal bundle to $M$ is a subbundle of $TN$ which picks out at each point the subspace which is normal to $TM$ in $TN$.
\end{itemize}

\begin{lem}\label{lem:vb-over-XxI}
  The restriction of a vector bundle $E\to X\times I$ to $\times \{0\}$ and $X\times \{1\}$ are isomorphic if $X$ is paracompact.
\end{lem}
\begin{cor}\label{cor:homotopies-of-vb}
  If $f:A\to B$ is a homotopy equivalence of vector bundles then the pullback $f^*$ induces a bijection $f^*:\Vect_K^n(B) \to \Vect_K^n(A)$ (remember that $\Vect^n_K(X)$ is the set of isomorphism classes of rank $n$ vector bundles with base field $K$ over $X$).
\end{cor}
\begin{example}
  Consider an embedding $S^{n-2}\hookrightarrow S^n$. Then the open tubular neighborhoods of $X^{n-2}$ are in bijection with points of the total space of rank 2 vector bundles over $S^{n-2}$.
\end{example}

\section{$K^0$}

$\Vect_{\mathbb C}^n(X)$ has the natural structure of an abelian monoid under the Whitney sum operation. This means we can complete it to a group, and this process is known as Groethendieck completion. For an arbitrary monoid it's this:
\begin{align*}
  \text{Gr}(M) = \mathbb Z \langle M \rangle / ([m] + [n] - [m + n], m,n \in M)
\end{align*}
and we define $K^0(X) = \operatorname{Gr}(\Vect^n_{\mathbb C}(X))$.
\end{document}
