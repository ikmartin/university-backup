\documentclass[hidelinks,11pt,dvipsnames]{article}

% font stuff
\linespread{1.1}

% xcolor commonly causes option clashes, this fixes that
\PassOptionsToPackage{dvipsnames,table}{xcolor}
\usepackage[tmargin=1in, bmargin=1in, lmargin=0.8in, rmargin=1in]{geometry}


% use the better epsilon
\renewcommand{\epsilon}{\varepsilon}

% suppress the warning page
\pdfsuppresswarningpagegroup=1

% enable synctex for inverse search
\synctex=1

% include packages
\usepackage{macrosabound,homework,math-env,quiver}
\usepackage{microtype}

% bibtex stuff
\usepackage[backend=biber,style=alphabetic,sorting=anyt]{biblatex}
\addbibresource{main.bib}

% colored text shortcuts
\newcommand{\blue}[1]{\color{MidnightBlue}{#1}}
\newcommand{\red}[1]{\textcolor{Mahogany}{#1}}
\newcommand{\green}[1]{\textcolor{ForestGreen}{#1}}

% use mathptmx pkg while using default mathcal font
\DeclareMathAlphabet{\mathcal}{OMS}{cmsy}{m}{n}

% fixes the positioning of subscripts in $$ $$
\renewcommand{\det}{\operatorname{det}}

\usetikzlibrary{positioning, arrows.meta}
\newcommand{\here}[2]{\tikz[remember picture]{\node[inner sep=0](#2){#1}}}


\usepackage{indentfirst}

% Title page stuff
\title{Fundamental Theorem of Arithmetic in the Tropics}
\date{Last Compiled: \today}
\author{Isaac Martin}

% start document
\begin{document}
\pagestyle{empty}
\maketitle
\newpage

\section{An attempted formulation of the FTA in the tropics}
This formulation only works for what I'm calling \textbf{irredundant} tropical polynomials. By this, I mean a tropical polynomial
\begin{align*}
    f = a_{i_n}\odot x^{\odot i_n} \oplus ... \oplus a_{i_0}
\end{align*}
such that the tropical polynomial $\hatf_{j}$ obtained by removing the monomial $a_{i_j}\oplus x^{\odot i_j}$ from $f$ is distinct from $f$ for each $j \in \{0..n\}$. If $f$ is not irredundant, then it is \textbf{redundant}. 

I'm not calling this a ``minimal'' tropical polynomial yet because I don't yet know if the irredundant presentaiton of $f$ minimizes the number of monomials needed or not. I believe it does, I just haven't shown that yet. Note also that this formulation mandates assigning subscripts to the powers of the monomials, as setting a coefficient to $\infty$ would make $f$ redundant.

Redundant is nice because of the following fact:


Anyways, here's an attempt at the Fundamental Theorem of Algebra.
\begin{thm}\label{thm:fta}
  Let $f$ be an irredundant tropical polynomial of degree $d$ so that
  \begin{align*}
    f = a_{i_n}\odot x^{\odot i_n} \oplus ... \oplus a_{i_0}
  \end{align*}
  where $0 = i_0 < i_1 < ... < i_n = d$. Then for all $x \in \bR\cup \{\infty\}$,
  \begin{align*}
    f(x) = a_{i_n} \bigodot_{j = 1}^n (x\oplus b_j)^{\odot m_j}
  \end{align*}
  where $m_j = i_j - i_{j-1}$ is the gap between subsequent integer powers and $b_j = \frac{a_{j-1} - a_j}{m_j}$.
\end{thm}
\begin{prf}
  Set $g(x) = a_{i_n} \bigodot_{j = 1}^n (x\oplus b_j)^{\odot m_j}$. Then we get
  \begin{align*}
    g(x) = a
  \end{align*}
\end{prf}
\end{document}
