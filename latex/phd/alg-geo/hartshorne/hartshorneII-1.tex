\documentclass[hidelinks,11pt,dvipsnames]{article}

% font stuff
\linespread{1.1}

% xcolor commonly causes option clashes, this fixes that
\PassOptionsToPackage{dvipsnames,table}{xcolor}
\usepackage[tmargin=1in, bmargin=1in, lmargin=0.8in, rmargin=1in]{geometry}


% use the better epsilon
\renewcommand{\epsilon}{\varepsilon}

% suppress the warning page
\pdfsuppresswarningpagegroup=1

% enable synctex for inverse search
\synctex=1

% include packages
\usepackage{macrosabound,homework,math-env,quiver}
\usepackage{microtype}

% bibtex stuff
\usepackage[backend=biber,style=alphabetic,sorting=anyt]{biblatex}
\addbibresource{main.bib}

% colored text shortcuts
\newcommand{\blue}[1]{\color{MidnightBlue}{#1}}
\newcommand{\red}[1]{\textcolor{Mahogany}{#1}}
\newcommand{\green}[1]{\textcolor{ForestGreen}{#1}}

% use mathptmx pkg while using default mathcal font
\DeclareMathAlphabet{\mathcal}{OMS}{cmsy}{m}{n}

% fixes the positioning of subscripts in $$ $$
\renewcommand{\det}{\operatorname{det}}

\usetikzlibrary{positioning, arrows.meta}
\newcommand{\here}[2]{\tikz[remember picture]{\node[inner sep=0](#2){#1}}}



\begin{document}
\pagestyle{empty}
	\LARGE
\begin{center}
	Problems from Hartshorne Chapter 2.2 \\
	\Large
	Isaac Martin \\
    Last compiled \today
\end{center}
\normalsize
\vspace{-2mm}
\hru
\begin{homework}[e]
	\prob Let $A$ be an abelian group and defined the \emph{constant presheaf} associated to $A$ on the topological space $X$ to be the preshseaf $U\mapsto A$ for all $U\neq \emptyset$, with restriction maps the identity. Show that the constant sheaf $\cA$ defined in the text is the sheaf associated to this presheaf.
	\begin{prf}
		Let $\cC$ be the constant sheaf on $X$, i.e. the sheaf defined as follows: for any open $U\subseteq X$, $\cC(U)$ is the group of all continuous maps of $U$ into $A$ (where $A$ is endowed with the discrete topology). Let $\cG$ be any other sheaf on $X$.

		Define $\theta:\cF\to \cC$ as follows. For an open set $U$, let $\theta(U):\cF(U) = A\to \cC(U)A$ send a point $a \in A$ to the constant map $(x \mapsto a)\in \cC(U)$.

		Now suppose we have some morphism $\alpha:\cF\to \cG$. We would like to define $\beta:\cC\to \cG$ such that $\beta\circ\theta = \alpha$.

		Fix an open subset $U \subseteq X$ and a section $f:U\to A$ of $\cC(U)$. Notice that $\{f^{-1}(a)\}_{a\in A}$ is an open cover of $U$ and $f|_{f^{-1}(a)} = (x\mapsto a) = \theta(U)(a)$ for all $a \in A$. Consider the collection $\{\alpha(U)(a)\}_{a\in A}$ of sections in $\cG(U)$. These satisfy the gluing compatibility condition, namely
		\begin{align*}
			\alpha(U)(a)|_{f^{-1}(a)\cap f^{-1}(b)} = \alpha(U)(b)|_{f^{-1}(a)\cap f^{-1}(b)}
		\end{align*}
		and hence there is some element $g_f \in \cG(U)$ such that $g_f|_{f^{-1}(a)} = \alpha(U)(a)|_{f^{-1}(a)}$ for all $a \in A$. We simply define $\beta(U)(f) = g_f$ to obtain a map $\beta(U):\cC(U)\to \cG(U)$. This satisfies the restriction requirements and hence $\beta$ is a map of schemes. Furthermore, if $f = \theta(U)(a)$ for some $a \in A$, then $f$ is the constant map $x \mapsto a$ and hence $f^{-1}(a) = U$, so $\beta(f) = \alpha(U)(a)$. This shows that $\alpha = \beta\circ \theta$, meaning $\cC$ satisfies the universal property of the sheaf associated to $\cF$.
	\end{prf}
	\prob $ $
	\begin{enumerate}[(a)]
		\item For any morphism of sheaves $\varphi:\cF\to \cG$ show that for each point $P$, $(\ker\varphi)_P = \ker(\varphi_P)$ and $(\im\varphi)_P = \im(\varphi_P)$
		\item Show that $\varphi$ is injective (respectively, surjective) if and only if the induced map on the stalks $\varphi_P$ is injective (respectively, surjective) for all $P$.
		\item Show that a sequence $...\to \cF^{i-1}\xrightarrow{\varphi^{i-1}} \cF^i \xrightarrow{\varphi^i}\cF^{i+1}\to ...$ of sheaves and morphisms is exact if and only if for each $P\in X$ the corresponding sequence of stalks is exact as a sequence of abelian groups.
	\end{enumerate}
	\begin{prf}$ $
		\begin{enumerate}[(a)]
			\item Recall that for any $V\subseteq X$ containing a point $P$ we have the diagram
				\begin{center}
					\begin{tikzcd}
						\cF(V) \arrow[r,"\varphi(V)"] \arrow[d,"\pi"]& \cG(V) \arrow[d,"\pi"] \\
						\cF_P \arrow[r,"\varphi_P"]&\cG_P
					\end{tikzcd}
				\end{center}
				Start with an element $(t,V) \in \ker (\varphi_P)$. Then $t$ is a section of $\cF(V)$ by definition and by commutativity of the diagram we have that $\pi(\varphi(V)(t)) = (\varphi(V)(t),V) = 0$ in $\cG_P$. This means that there is some open neighborhood $W\subset V$ of $P$ such that $\varphi(U)(t)|_W = 0$ by the equivalence relation on $\cG_P$, and since $\varphi(U)(t)|_W = \varphi(W)(t)$ we have that $\varphi(W)(t|_W) = 0$. Hence $t|_W = 0$ and so $t \in \ker\varphi(W)$. Hence $(t|_W,W) \in (\ker \varphi)_P$, and because $(t|_W,W)$ and $(t,V)$ represent the same element in $\ker (\varphi_P)$, this shows the inclusion $\ker(\varphi_P) \subseteq (\ker \varphi)_P$.

				For the other inclusion, take an element $(t,V) \in (\ker\varphi)_P$. This means that $t \in (\ker\varphi)(V) = \ker(\varphi(V))$ and hence $\varphi(V)(t) = 0$ in $\cG(V)$. Composing with $\pi$ gives $\pi(\varphi(V)(t)) = (\varphi(V)(t),V) = 0$ in $\cG_P$. By commutativity, $\pi((t,V)) = (t,V)\in \cF_P$ maps to $0$ under $\varphi_P$, so $(t,V)\in \ker(\varphi_P)$. This gives us the other inclusion.

				\bigskip

				Now let's consider $\im(\varphi)$.
		\end{enumerate}
	\end{prf}
	\prob $ $
	\begin{enumerate}[(a)]
		\item  Let $\varphi:\cF\to \cG$ be a morphism of sheaves on $X$. Show that $\varphi$ is surjective if and only if the following condition holds: for every open set $U\subseteq X$, and for every $s \in \cG(U)$, there is a covering $\{U_i\}$ of $U$ and there are elements $t_i \in \cF(U_i)$ such that $\varphi(t_i) = s|_{U_i}$ for all $i$.
		\item Give an example of a surjective morphism of sheaves $\varphi:\cF\to \cG$ and an open set $U$ such that $\varphi(U):\cF(U)\to \cG(U)$ is not surjective.
	\end{enumerate}
\end{homework}
\end{document}
