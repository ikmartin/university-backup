%%%%%%%%%%%%%%%%%%%%%%%%%%%%%%%%%%%%%%%%%
% Tufte-Style Book (Documentation Template)
% LaTeX Template
% Version 1.0 (5/1/13)
%
% This template has been downloaded from:
% http://www.LaTeXTemplates.com
%
% Original author:
% The Tufte-LaTeX Developers (tufte-latex.googlecode.com)
%
% License:
% Apache License (Version 2.0)
%
% IMPORTANT NOTE:
% In addition to running BibTeX to compile the reference list from the .bib
% file, you will need to run MakeIndex to compile the index at the end of the
% document.
%
%%%%%%%%%%%%%%%%%%%%%%%%%%%%%%%%%%%%%%%%%

%----------------------------------------------------------------------------------------
%	PACKAGES AND OTHER DOCUMENT CONFIGURATIONS
%----------------------------------------------------------------------------------------

\documentclass[11pt, ragged-right]{tufte-book} % Use the tufte-book class which in turn uses the tufte-common class

%%%%%%%%%%%%%%%%%%%%%%%%%%%%%%%%
% SETTINGS
%%%%%%%%%%%%%%%%%%%%%%%%%%%%%%%%
\hypersetup{colorlinks} % Comment this line if you don't wish to have colored links

\setcounter{tocdepth}{2}
\setcounter{secnumdepth}{1}

\usepackage{fancyvrb} % Allows customization of verbatim environments
\fvset{fontsize=\normalsize} % The font size of all verbatim text can be changed here

%%%%%%%%%%%%%%%%%%%%%%%%%%%%%%%%
%%% CHANGE CHAPTER FORMATTING
%%%%%%%%%%%%%%%%%%%%%%%%%%%%%%%%
\makeatletter
\titleformat{\chapter}%
  {}% format applied to label+text
  {\itshape\Huge\chaptertitlename~\thechapter:}% label
  {12pt}% horizontal separation between label and title body
  {\Huge\rmfamily\itshape}% before the title body
\makeatother
\makeatletter
\titleformat{\section}%
  {}% format applied to label+text
  {\huge\rmfamily\itshape \S \arabic{section}:}% label
  {9pt}% horizontal separation between label and title body
  {\huge\rmfamily\itshape}% before the title body
\makeatother

%%%%%%%%%%%%%%%%%%%%%%%%%%%%%%%%
%%% PACKAGES TO INCLUDE
%%%%%%%%%%%%%%%%%%%%%%%%%%%%%%%%
\usepackage{microtype} % Improves character and word spacing
\usepackage{booktabs} % Better horizontal rules in tables
\usepackage{xspace} % Used for printing a trailing space better than using a tilde (~) using the \xspace command

%%%%%%%%%%%%%%%%%%%%%%%%%%%%%%%%
%%% FIGURES
%%%%%%%%%%%%%%%%%%%%%%%%%%%%%%%%
\usepackage{graphicx} % Needed to insert images into the document
\graphicspath{{images/}} % Sets the default location of pictures
\setkeys{Gin}{width=\linewidth,totalheight=\textheight,keepaspectratio} % Improves figure scaling

%%%%%%%%%%%%%%%%%%%%%%%%%%%%%%%%%%%%
%%% CUSTOM COMMANDS AND ENVIRONMENTS
%%%%%%%%%%%%%%%%%%%%%%%%%%%%%%%%%%%%

\newcommand{\blankpage}{\newpage\hbox{}\thispagestyle{empty}\newpage} % Command to insert a blank page
\newcommand{\hlred}[1]{\textcolor{Maroon}{#1}} % Print text in maroon
\newcommand{\hairsp}{\hspace{1pt}} % Command to print a very short space
\newcommand{\ie}{\textit{i.\hairsp{}e.}\xspace} % Command to print i.e.
\newcommand{\eg}{\textit{e.\hairsp{}g.}\xspace} % Command to print e.g.
\newcommand{\na}{\quad--} % Used in tables for N/A cells

%%%%%%%%%%%%%%%%%%%%%%%%%%%%%%%%%%%%
%%% PACKAGE LOADS
%%%%%%%%%%%%%%%%%%%%%%%%%%%%%%%%%%%%
\usepackage{macrosabound, math-env}
\usetikzlibrary{arrows}
