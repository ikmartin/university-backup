\documentclass[hidelinks,11pt,dvipsnames]{article}

% font stuff
\linespread{1.1}

% xcolor commonly causes option clashes, this fixes that
\PassOptionsToPackage{dvipsnames,table}{xcolor}
\usepackage[tmargin=1in, bmargin=1in, lmargin=0.8in, rmargin=1in]{geometry}


% use the better epsilon
\renewcommand{\epsilon}{\varepsilon}

% suppress the warning page
\pdfsuppresswarningpagegroup=1

% enable synctex for inverse search
\synctex=1

% include packages
\usepackage{macrosabound,homework,math-env,quiver}
\usepackage{microtype}

% bibtex stuff
\usepackage[backend=biber,style=alphabetic,sorting=anyt]{biblatex}
\addbibresource{main.bib}

% colored text shortcuts
\newcommand{\blue}[1]{\color{MidnightBlue}{#1}}
\newcommand{\red}[1]{\textcolor{Mahogany}{#1}}
\newcommand{\green}[1]{\textcolor{ForestGreen}{#1}}

% use mathptmx pkg while using default mathcal font
\DeclareMathAlphabet{\mathcal}{OMS}{cmsy}{m}{n}

% fixes the positioning of subscripts in $$ $$
\renewcommand{\det}{\operatorname{det}}

\usetikzlibrary{positioning, arrows.meta}
\newcommand{\here}[2]{\tikz[remember picture]{\node[inner sep=0](#2){#1}}}



\begin{document}
\noindent{\textsc{Problem 2.13}}
\begin{proof}$ $
  \begin{enumerate}[(a)]
    \item Suppose that $X$ is a Noetherian space, $U$ is an open set and $\{V_\alpha\}_{\alpha \in A}$ is an open cover. Let $W_n = \bigcup_{i=1}^n V_{\alpha_i}$ where $\alpha_i \neq \alpha_j$ whenever $i \neq j$. Further define $F_i = U \setminus W_i$ to be the complement of $W_i$ in $U$. Then $U = F_0 \supsetneq F_1 \supsetneq ...$ is a descending chain of a closed subsets and hence must terminate at some $F_n = \emptyset$. This then implies that $V_n = U$.

    Now suppose that $X$ is not a Noetherian space, meaning there exists some infinite descending chain of closed subsets $X = F_0 \supsetneq F_1 \supsetneq F_2 \supsetneq ...$. Define the closed set $F = \bigcap F_i$ and the open set $U = X \setminus F$, along with $U_i = X\setminus F_i$. Then $\{U_i\}$ forms a cover of $U$ but has no finite subcover.
    \item Let $X = \Spec A$ be an affine scheme. We first show it is quasi-compact, so take some open cover $\{U_\alpha\}_{\cA}$ of $X$. Without loss of generality we may take all these to be basic opens, so instead take $\{D(f_i)\}_{i\in \cI}$ to be the open cover. This means
      \begin{align*}
        X = D(1) \subset \bigcup D(f_i) &\iff \emptyset = V(1) \supset \bigcap V((f_i)) = V\left(\sum (f_i)\right)
      \end{align*}
      meaning that $1 \in \sqrt{\sum (f_i)} \implies 1 \in \sum (f_i)$, i.e. $1 = a_1f_{i_1} + ... + a_n f_{i_n}$ for some $a_1,...,a_n \in A$. This then implies that $D(f_{i_1})\cup ... \cup D(f_{i_n}) = X$, so $X$ is quasi-coherent.

      \bigskip

      To see that $X$ isn't necessarily, Noetherian, take any non-Noetherian ring $A$ and an ascending chain $\fraka_0 \subsetneq\fraka_1 \subsetneq ...$ of ideals which doesn't terminate. Then $V(\fraka_0) \supsetneq V(\fraka_1) \supsetneq ...$ is a descending chain of closed subsets which never terminates.
  \end{enumerate}
\end{proof}
\end{document}
