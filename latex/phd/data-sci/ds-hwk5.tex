\documentclass[hidelinks,11pt,dvipsnames]{article}

% font stuff
\linespread{1.1}

% xcolor commonly causes option clashes, this fixes that
\PassOptionsToPackage{dvipsnames,table}{xcolor}
\usepackage[tmargin=1in, bmargin=1in, lmargin=0.8in, rmargin=1in]{geometry}


% use the better epsilon
\renewcommand{\epsilon}{\varepsilon}

% suppress the warning page
\pdfsuppresswarningpagegroup=1

% enable synctex for inverse search
\synctex=1

% include packages
\usepackage{macrosabound,homework,math-env,quiver}
\usepackage{microtype}

% bibtex stuff
\usepackage[backend=biber,style=alphabetic,sorting=anyt]{biblatex}
\addbibresource{main.bib}

% colored text shortcuts
\newcommand{\blue}[1]{\color{MidnightBlue}{#1}}
\newcommand{\red}[1]{\textcolor{Mahogany}{#1}}
\newcommand{\green}[1]{\textcolor{ForestGreen}{#1}}

% use mathptmx pkg while using default mathcal font
\DeclareMathAlphabet{\mathcal}{OMS}{cmsy}{m}{n}

% fixes the positioning of subscripts in $$ $$
\renewcommand{\det}{\operatorname{det}}

\usetikzlibrary{positioning, arrows.meta}
\newcommand{\here}[2]{\tikz[remember picture]{\node[inner sep=0](#2){#1}}}



\usepackage{capt-of}
\usepackage{tikz}
\usepackage{listings}
\usetikzlibrary{positioning,calc,intersections,through,backgrounds, shapes.geometric, decorations.markings,arrows}

\def\sset{\subseteq}
\def\iso{\cong}
\def\gend#1{\langle #1\rangle}

\newcommand{\rightoverleftarrow}{%
  \mathrel{\vcenter{\mathsurround0pt
    \ialign{##\crcr
      \noalign{\nointerlineskip}$\longrightarrow$\crcr
      \noalign{\nointerlineskip}$\longleftarrow$\crcr
    }%
  }}%
}

\newcommand\makesphere{} % just for safety
\def\makesphere(#1)(#2)[#3][#4]{%
  % Synopsis
  % \makesphere[draw options](center)(initial angle:final angle:radius)
  \shade[ball color = #3, opacity = #4] #1 circle (#2);
  \draw #1 circle (#2);
  \draw ($#1 - (#2, 0)$) arc (180:360:#2 and 3*#2/10);
  \draw[dashed] ($#1 + (#2, 0)$) arc (0:180:#2 and 3*#2/10);
}
% same thing as makesphere but places white background behind
\newcommand\altmakesphere{} % just for safety
\def\altmakesphere(#1)(#2)(#3)[#4][#5]{%
  % Synopsis
  % \make sphere[draw options](center)(initial angle:final angle:radius)
  \draw [fill=white!30] #1 circle (#2);
  \shade[ball color = #4, opacity = #5] #1 circle (#2);
  \draw #1 circle (#2);
  \draw ($#1 - (#2, 0)$) arc (180:360:#2 and 3*#2/10);
  \draw[dashed] ($#1 + (#2, 0)$) arc (0:180:#2 and 3*#2/10);
  \node at #1 {#3};
}

\begin{document}
% set section number to 1
% fixes theorem numbering without need to have a section title
\setcounter{section}{1}

\pagestyle{empty}
	\LARGE
\begin{center}
	Foundations of Data Science and Machine Learning -- \emph{Homework 5}\\
	\Large
	Isaac Martin \\
    Last compiled \today
\end{center}
\normalsize
\vspace{-4mm}
\hru

\begin{homework}[e]
  \prob Suppose $\bfA$ is a $n\times d$ full-rank matrix, with $n < d$, and fix $\bfb\in \bR^n$. Consider minimizing the least squares objective $F(\bfx) = \|\bfA\bfx - \bfb\|^2_2$ Note that in this setting, the solution space $\cS = \{x : \bfA\bfx = \bb\}$ is an affine subspace of $\bR^d$. We use gradient descent with constant step-size:
  \begin{align*}
    \bfx = \bfx_{k-1} - \eta \nabla F(\bfx_{k-1}).
  \end{align*}
  \begin{enumerate}[(a)]
    \item Give an upper bound for the step-size $\eta$ such that gradient descent is guaranteed to converge for $\eta$ below this threshold.
    \item Suppose taht gradient descent is initialized at $\bfx_0 = 0$. Show that when gradient descent converges, it must converge to the least-norm solution $\bfx^* = \operatorname{argmin}_{\bfx \in \cS} \|\bfx\|^2_2$.
  \end{enumerate}

  \prob Let $f:\bR^d \to \bR$ be a differentiable function. It satisfies the PL-inequality if there exists a constant $\mu > 0$ such that for all $w \in \bR^d$ it holds
  \begin{align*}
    \frac{1}{2}\|\nabla f(w)\|^2_2 \geq \mu(f(w) - f^*).
  \end{align*}
  By contrast we say $f$ is \emph{invex} if there exists a function $\eta:\bR^d\times \bR^d\to \bR^d$ such that for all $x, y\in \bR^d$ it holds
  \begin{align*}
    f(y) \geq f(x) + \nabla f(x)^\top \eta(x,y).
  \end{align*}
  \begin{enumerate}[(a)]
    \item Show that if $f$ satisfies the PL-inequality then $f$ is invex.
    \item Show that any stationary point of an invex function is a global minimizer.
  \end{enumerate}
  \begin{prf}$ $
    \begin{enumerate}[(a)]
      \item 
      \item A point $\bfx$ is a stationary point of $f$ if $\nabla f(x) = 0$. If $f$ is invex, then we get
        \begin{align*}
          f(y) \geq f(x) + \nabla f(x)^top \eta(x,y) = f(x)
        \end{align*}
        for all $y\in \bR^d$. Hence any stationary point of $f$ is a global minima. Combining with part (a) we see that any function which satisfies the PL-inequality is easily optimized.
    \end{enumerate}
  \end{prf}
\end{homework}
\newpage
\begin{verbatim}  
\end{verbatim}
\end{document}
