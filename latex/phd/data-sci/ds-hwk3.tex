\documentclass[hidelinks,11pt,dvipsnames]{article}

% font stuff
\linespread{1.1}

% xcolor commonly causes option clashes, this fixes that
\PassOptionsToPackage{dvipsnames,table}{xcolor}
\usepackage[tmargin=1in, bmargin=1in, lmargin=0.8in, rmargin=1in]{geometry}


% use the better epsilon
\renewcommand{\epsilon}{\varepsilon}

% suppress the warning page
\pdfsuppresswarningpagegroup=1

% enable synctex for inverse search
\synctex=1

% include packages
\usepackage{macrosabound,homework,math-env,quiver}
\usepackage{microtype}

% bibtex stuff
\usepackage[backend=biber,style=alphabetic,sorting=anyt]{biblatex}
\addbibresource{main.bib}

% colored text shortcuts
\newcommand{\blue}[1]{\color{MidnightBlue}{#1}}
\newcommand{\red}[1]{\textcolor{Mahogany}{#1}}
\newcommand{\green}[1]{\textcolor{ForestGreen}{#1}}

% use mathptmx pkg while using default mathcal font
\DeclareMathAlphabet{\mathcal}{OMS}{cmsy}{m}{n}

% fixes the positioning of subscripts in $$ $$
\renewcommand{\det}{\operatorname{det}}

\usetikzlibrary{positioning, arrows.meta}
\newcommand{\here}[2]{\tikz[remember picture]{\node[inner sep=0](#2){#1}}}



\usepackage{capt-of}
\usepackage{tikz}
\usepackage{listings}
\usetikzlibrary{positioning,calc,intersections,through,backgrounds, shapes.geometric, decorations.markings,arrows}

\def\sset{\subseteq}
\def\iso{\cong}
\def\gend#1{\langle #1\rangle}

\newcommand{\rightoverleftarrow}{%
  \mathrel{\vcenter{\mathsurround0pt
    \ialign{##\crcr
      \noalign{\nointerlineskip}$\longrightarrow$\crcr
      \noalign{\nointerlineskip}$\longleftarrow$\crcr
    }%
  }}%
}

\newcommand\makesphere{} % just for safety
\def\makesphere(#1)(#2)[#3][#4]{%
  % Synopsis
  % \makesphere[draw options](center)(initial angle:final angle:radius)
  \shade[ball color = #3, opacity = #4] #1 circle (#2);
  \draw #1 circle (#2);
  \draw ($#1 - (#2, 0)$) arc (180:360:#2 and 3*#2/10);
  \draw[dashed] ($#1 + (#2, 0)$) arc (0:180:#2 and 3*#2/10);
}
% same thing as makesphere but places white background behind
\newcommand\altmakesphere{} % just for safety
\def\altmakesphere(#1)(#2)(#3)[#4][#5]{%
  % Synopsis
  % \make sphere[draw options](center)(initial angle:final angle:radius)
  \draw [fill=white!30] #1 circle (#2);
  \shade[ball color = #4, opacity = #5] #1 circle (#2);
  \draw #1 circle (#2);
  \draw ($#1 - (#2, 0)$) arc (180:360:#2 and 3*#2/10);
  \draw[dashed] ($#1 + (#2, 0)$) arc (0:180:#2 and 3*#2/10);
  \node at #1 {#3};
}

\begin{document}
% set section number to 1
% fixes theorem numbering without need to have a section title
\setcounter{section}{1}

\pagestyle{empty}
	\LARGE
\begin{center}
	Foundations of Data Science and Machine Learning -- \emph{Homework 3}\\
	\Large
	Isaac Martin \\
    Last compiled \today
\end{center}
\normalsize
\vspace{-4mm}
\hru

\begin{homework}[e]
  \prob[\textsc{Exercise 3.}] Find the mapping $\varphi(\bfx)$ that gives rise to the polynomial kernel \begin{align*}
    K(\bfx, \bfy) = (x_1x_2 + y_1y_2)^2.
  \end{align*}
  \begin{prf}
    Consider the map $\varphi:\bR^2 \to \bR^3$ defined $\varphi(x_1,x_2) = (x_1^2, x_2^2, \sqrt{2}x_1x_2)$. Interestingly, this is similar to the map one considers from a polynomial ring $R[x_1,x_2]$ to its $2^{\text{nd}}$ Veronese subring $R[x_1^2,x_1x_2,x_2^2]$. We then have that
    \begin{align*}
      \varphi(\bfx)\cdot \varphi(\bfy)^T
        &= (x_1^2,x_2^2,\sqrt{2}x_1x_2) \cdot (y_1^2,y_2^2, \sqrt{2}y_1y_2) \\
        &= x_1^2y_1^2 + x_2^2y_2^2 + 2x_1y_1x_2y_2 \\
        &= (x_1y_1 + x_2y_2)^2,
    \end{align*}
    hence $\varphi$ gives rise to the desired kernel.
  \end{prf}
\end{homework}
\end{document}
